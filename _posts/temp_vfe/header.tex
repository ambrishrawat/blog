\documentclass[twoside]{article}
\usepackage{amsmath,amssymb,amsthm,graphicx}
\usepackage{epsfig}
\usepackage[authoryear]{natbib}


\newcommand{\ind}[1]{1_{#1}} % Indicator function
\newcommand{\pr}{P} % Generic probability
\newcommand{\ex}{E} % Generic expectation
\newcommand{\var}{\textrm{Var}}
\newcommand{\cov}{\textrm{Cov}}
\newcommand{\sgn}{\textrm{sgn}}
\newcommand{\sign}{\textrm{sign}}
\newcommand{\kl}{\textrm{KL}} 
\newcommand{\abs}[1]{|{#1}|}

% Theorem-like declarations
\theoremstyle{plain}
\newtheorem{theorem}{Theorem}
\newtheorem{corollary}[theorem]{Corollary}
\newtheorem{lemma}[theorem]{Lemma}

\theoremstyle{definition}
\newtheorem{definition}[theorem]{Definition}
\newtheorem{example}[theorem]{Example}

\theoremstyle{remark}
\newtheorem{remark}[theorem]{Remark}

\renewcommand{\S}{\Sigma}
\renewcommand{\L}{\Lambda}
\renewcommand{\[}{\begin{equation}}
\renewcommand{\]}{\end{equation}}
\renewcommand{\b}{\backslash}
\newcommand{\g}{\,\vert\,}
\newcommand{\tr}{\mathrm{tr}}
\newcommand{\diag}{\mathrm{diag}}
\newcommand{\bea}{\begin{eqnarray}}
\newcommand{\eea}{\end{eqnarray}}
\newcommand{\hx}{\hat{x}}
\newcommand{\hxi}{\hat{\xi}}
\newcommand{\Var}{\mathrm{Var}}
\newcommand{\Cov}{\mathrm{Cov}}
\newcommand{\prop}{\propto}
\newcommand{\deq}{:=}

\newcommand{\EE}{\mathbb{E}}
\newcommand{\II}{\mathbb{I}}
\newcommand{\R}{\mathbb{R}}
\newcommand{\PP}{\mathbb{P}}

\newcommand{\La}{\mathcal{L}}

\newcommand{\n}{\mathcal{N}}

\newcommand{\bx}{\mathbf{x}}
\newcommand{\bX}{\mathbf{X}}
\newcommand{\by}{\mathbf{y}}
\newcommand{\bs}{\mathbf{s}}
\newcommand{\bn}{\mathbf{n}}
\newcommand{\br}{\mathbf{r}}
\newcommand{\bt}{\mathbf{t}}

\newcommand{\fig}[1]{Figure~\ref{fig:#1}}
\newcommand{\chap}[1]{Chapter~\ref{chap:#1}}
\newcommand{\mysec}[1]{Section~\ref{sec:#1}}
\newcommand{\app}[1]{Appendix~\ref{sec:#1}}
\newcommand{\eq}[1]{Eq.~(\ref{eq:#1})}
\newcommand{\eqs}[1]{Eqs.~(\ref{eq:#1})}
\newcommand{\eqss}[1]{(\ref{eq:#1})}
\newcommand{\thm}[1]{Theorem~\ref{thm:#1}}

\newcommand{\indep}{{\;\bot\!\!\!\!\!\!\bot\;}}
\newcommand{\eps}{\varepsilon}

\newcommand{\one}{1}
\newcommand{\Dir}{{\rm Dir}}
\newcommand{\Mult}{{\rm Mult}}
\newcommand{\Bin}{{\rm Bin}}
\newcommand{\Ga}{{\rm Ga}}
\newcommand{\IG}{{\rm IG}}
\newcommand{\InvGa}{{\rm IG}}
\newcommand{\Chisquare}{\Chi^2}
\newcommand{\St}{{\rm St}}
\newcommand{\Beta}{{\rm Beta}}
\newcommand{\iid}{i.i.d.}
\newcommand{\Eta}{{\cal N}}
\newcommand{\Ber}{{\rm Ber}}

\DeclareMathOperator*{\BP}{BP}
\DeclareMathOperator*{\DP}{DP}
\DeclareMathOperator*{\GP}{GP}
\DeclareMathOperator*{\BeP}{BeP}

% Caligraphic alphabet
\newcommand{\calr}{\mathcal{R}} % only because \cr already taken
\newcommand{\ca}{\mathcal{A}} \newcommand{\cb}{\mathcal{B}} \newcommand{\cc}{\mathcal{C}} \newcommand{\cd}{\mathcal{D}} \newcommand{\ce}{\mathcal{E}} \newcommand{\cf}{\mathcal{F}} \newcommand{\cg}{\mathcal{G}} \newcommand{\ch}{\mathcal{H}} \newcommand{\ci}{\mathcal{I}} \newcommand{\cj}{\mathcal{J}} \newcommand{\ck}{\mathcal{K}} \newcommand{\cl}{\mathcal{L}} \newcommand{\cm}{\mathcal{M}} \newcommand{\cn}{\mathcal{N}} \newcommand{\co}{\mathcal{O}} \newcommand{\cp}{\mathcal{P}} \newcommand{\cq}{\mathcal{Q}} \newcommand{\cs}{\mathcal{S}} \newcommand{\ct}{\mathcal{T}} \newcommand{\cu}{\mathcal{U}} \newcommand{\cv}{\mathcal{V}} \newcommand{\cw}{\mathcal{W}} \newcommand{\cx}{\mathcal{X}} \newcommand{\cy}{\mathcal{Y}} \newcommand{\cz}{\mathcal{Z}}

% Convergence
\newcommand{\convd}{\stackrel{d}{\longrightarrow}} % convergence in distribution/law/measure
\newcommand{\convp}{\stackrel{P}{\longrightarrow}} % convergence in probability
\newcommand{\convas}{\stackrel{\textrm{a.s.}}{\longrightarrow}} % convergence almost surely
\newcommand{\convr}{\stackrel{r}{\longrightarrow}} % convergence in r^{th} mean

\newcommand{\eqd}{\stackrel{d}{=}} % equal in distribution/law/measure
\newcommand{\argmax}{\mathop{\mathrm{argmax}}}
\newcommand{\argmin}{\mathop{\mathrm{argmin}}}
\newcommand{\conv}{\textrm{conv}} % for denoting the convex hull



\setlength{\oddsidemargin}{0.25 in}
\setlength{\evensidemargin}{-0.25 in}
\setlength{\topmargin}{-0.6 in}
\setlength{\textwidth}{6.5 in}
\setlength{\textheight}{8.5 in}
\setlength{\headsep}{0.75 in}
\setlength{\parindent}{0 in}
\setlength{\parskip}{0.1 in}

\newcommand{\lecture}[4]{
   \pagestyle{myheadings}
   \thispagestyle{plain}
   \newpage
   \setcounter{page}{1}
   \noindent
   \begin{center}
   \framebox{
      \vbox{\vspace{2mm}
    \hbox to 6.28in { {\bf 90\%Humour - Eigentropy \hfill Blog Date: #4} }
       \vspace{6mm}
       \hbox to 6.28in { {\Large \hfill #1  \hfill}  }
       \vspace{6mm}
       \hbox to 6.28in { {\it Author: #2 \hfill} }
      \vspace{2mm}}
   }
   \end{center}
   \markboth{#1}{#1}
   \vspace*{4mm}
}

% Local Macros Put your favorite macros here that don't appear in
% stat-macros.tex.  We can eventually incorporate them into
% stat-macros.tex if they're of general use.

\begin{document}

\lecture{Variational Methods}{Ambrish Rawat}{}{February 12, 2017}

\section{Why use VI?}

- Its grounded in statistical physics

Variational Inference is viserally gratifying because of its grounding in statiscal physics. The marginal-likelihood, $\log p(D\vert m)$, differs from the system's free-energy $ \mathbb{E}_{q_{\lambda}(\theta)}[\log\frac{p(\theta, D\vert m)}{q_{\lambda}(\theta)}]$, which in simplified form is the sum of energy and entropy $\mathbb{E}_{q_{\lambda}(\theta)}[\log p(\theta, D\vert m)] + H(q)$.

- Its intergration simplified to optimisation

VI owes its popularity to reducing an intractable integral to an optimisation problem which involves derivative computations.

- It preserves the Bayesian modelling principles

With parametric and analytical form of approximate posterior, VI compromises the least on uncertainty representation.

- It can be coupled or extended with other (mean-field) apprxoimations

Guding principles in VI don't mandate a particular choice of $q_{\lambda}(\theta)$. In principle, an optimal choice of q can be obtained within any family of distributions. One extended approximation which is known to go well wioth VI, is the mean-field approximation.

- Why not use VI?

KL divergence is not symmetric. In fact, minimising the reverse-KL leads to other interesting properties which is found to be useful in another approximate inference scheme - Expectation Propagation. Due to a specific chocie of similarity measure, VI has no guraantees that the obtained approximate posterior will have desired properties.

- What are the qualitative propoerties of the obtained $q_{\lambda^*}(\theta)$

In high-dimensional spaces and multi-modal exact posteriors, VI often fails. The choice of KL-divergence results can result in an approximate posterior where the the optimal apprxoimationm, $q_{\lambda^*}(\theta)$ distributes its mass in a counter-intuitive fashion. For instance, it could place a high-mass on regions where the orignal distribution had a low or zero mass.

- Are all the uncertainties propagated?

As VI is often extended with a new layer of structure approximation like mean-field, it often fails to propagate uncertainties.



\end{document}



