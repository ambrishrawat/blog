\documentclass[twoside]{article}
\usepackage{amsmath,amssymb,amsthm,graphicx}
\usepackage{epsfig}
\usepackage[authoryear]{natbib}

\input{stat-macros}

\setlength{\oddsidemargin}{0.25 in}
\setlength{\evensidemargin}{-0.25 in}
\setlength{\topmargin}{-0.6 in}
\setlength{\textwidth}{6.5 in}
\setlength{\textheight}{8.5 in}
\setlength{\headsep}{0.75 in}
\setlength{\parindent}{0 in}
\setlength{\parskip}{0.1 in}

\newcommand{\lecture}[4]{
   \pagestyle{myheadings}
   \thispagestyle{plain}
   \newpage
   \setcounter{page}{1}
   \noindent
   \begin{center}
   \framebox{
      \vbox{\vspace{2mm}
    \hbox to 6.28in { {\bf 90\%Humour - Eigentropy \hfill Blog Date: #4} }
       \vspace{6mm}
       \hbox to 6.28in { {\Large \hfill #1  \hfill}  }
       \vspace{6mm}
       \hbox to 6.28in { {\it Author: #2 \hfill} }
      \vspace{2mm}}
   }
   \end{center}
   \markboth{#1}{#1}
   \vspace*{4mm}
}

% Local Macros Put your favorite macros here that don't appear in
% stat-macros.tex.  We can eventually incorporate them into
% stat-macros.tex if they're of general use.

\begin{document}

\lecture{Variational Methods}{Ambrish Rawat}{}{February 12, 2017}

\section{Why use VI?}

- Its grounded in statistical physics

Variational Inference is viserally gratifying because of its grounding in statiscal physics. The marginal-likelihood, $\log p(D\vert m)$, differs from the system's free-energy $ \mathbb{E}_{q_{\lambda}(\theta)}[\log\frac{p(\theta, D\vert m)}{q_{\lambda}(\theta)}]$, which in simplified form is the sum of energy and entropy $\mathbb{E}_{q_{\lambda}(\theta)}[\log p(\theta, D\vert m)] + H(q)$.

- Its intergration simplified to optimisation

VI owes its popularity to reducing an intractable integral to an optimisation problem which involves derivative computations.

- It preserves the Bayesian modelling principles

With parametric and analytical form of approximate posterior, VI compromises the least on uncertainty representation.

- It can be coupled or extended with other (mean-field) apprxoimations

Guding principles in VI don't mandate a particular choice of $q_{\lambda}(\theta)$. In principle, an optimal choice of q can be obtained within any family of distributions. One extended approximation which is known to go well wioth VI, is the mean-field approximation.

- Why not use VI?

KL divergence is not symmetric. In fact, minimising the reverse-KL leads to other interesting properties which is found to be useful in another approximate inference scheme - Expectation Propagation. Due to a specific chocie of similarity measure, VI has no guraantees that the obtained approximate posterior will have desired properties.

- What are the qualitative propoerties of the obtained $q_{\lambda^*}(\theta)$

In high-dimensional spaces and multi-modal exact posteriors, VI often fails. The choice of KL-divergence results can result in an approximate posterior where the the optimal apprxoimationm, $q_{\lambda^*}(\theta)$ distributes its mass in a counter-intuitive fashion. For instance, it could place a high-mass on regions where the orignal distribution had a low or zero mass.

- Are all the uncertainties propagated?

As VI is often extended with a new layer of structure approximation like mean-field, it often fails to propagate uncertainties.



\end{document}



